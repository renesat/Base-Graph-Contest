% Created 2020-10-10 Сб 19:39
% Intended LaTeX compiler: xelatex
\documentclass[14pt]{extarticle}
                

\usepackage{longtable}
\usepackage{wrapfig}
\usepackage{rotating}
\usepackage[normalem]{ulem}
\usepackage{amsmath}
\usepackage{breqn}
\usepackage{textcomp}
\usepackage{amssymb}
\usepackage{capt-of}
\usepackage{hyperref}
\usepackage{minted}
\usepackage{polyglossia}
\setmainlanguage{russian}
\setotherlanguage{english}
\setkeys{russian}{babelshorthands=true}
\PolyglossiaSetup{russian}{indentfirst=true}
\usepackage{fontspec}
\setmainfont{Liberation Serif}
\usepackage{minted}
\usepackage[left=4cm,right=4cm, top=2cm,bottom=2cm,bindingoffset=0cm]{geometry}
\usepackage{xcolor}
\PassOptionsToPackage{final}{graphicx}
\usepackage{caption}
\usepackage{subcaption}
\usepackage{wrapfig}
\usepackage{array}
\usepackage{multirow}
\usepackage{makecell}
\definecolor{friendlybg}{HTML}{f0f0f0}
\date{}
\title{Задачи на графы}
\hypersetup{
 pdfauthor={},
 pdftitle={Задачи на графы},
 pdfkeywords={},
 pdfsubject={},
 pdfcreator={Emacs 27.1 (Org mode 9.4)}, 
 pdflang={Russian}}
\begin{document}



\section*{Задача 1}
\label{sec:orgec1537c}
Задан неорентированный граф.
Граф задан количеством вершин \(n\) (\(1 \leq n \leq 105\)) и
списком пар вершин образующих рёбра.

Необходимо найти степени всех
вершин.
В результате должен получится список
степеней вершин по порядку номера для заданного графа.

Граф без петель и кратных рёбер.

\section*{Задача 2}
\label{sec:org541444b}
Необходимо построить \(k\text{-регулярный}\) неориентированный граф из \(n\) вершин.
В результате должен быть получен граф заданный списком из пар вершин образующих
рёбра.
Если это невозможно, то необходимо вернуть None.

\section*{Задача 3}
\label{sec:org75c2ce4}
Заданы 2 числа: степень исхода для каждой вершины (\(d_{1}\)) и степень выхода (\(d_{2}\)).
Необходимо построить наименьший непустой ориентированнный граф со соответствующими
степенями исхода и входа.

В результате должен быть получен граф заданный количеством вершин и списком дуг.

\section*{Задача 4}
\label{sec:org1f5a5f7}
Задан неорентированный граф.
Граф задан количеством вершин \(n\) (\(1 \leq n \leq 105\)) и
списком пар вершин образующих рёбра.
Так же задан список вершин.

Необходимо проверить, что заданные вершины образуют точно одну или более
компонент связности заданного графа.
Тоесть необходимо проверить можно ли выбрать подмножество компонент связности так,
что заданные вершины являются всеми вершинами этого подмножества компонент (и только ими).
\end{document}
