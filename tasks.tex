% Created 2020-10-08 Чт 23:55
% Intended LaTeX compiler: xelatex
\documentclass{extarticle}
                

\usepackage{longtable}
\usepackage{wrapfig}
\usepackage{rotating}
\usepackage[normalem]{ulem}
\usepackage{amsmath}
\usepackage{breqn}
\usepackage{textcomp}
\usepackage{amssymb}
\usepackage{capt-of}
\usepackage{hyperref}
\usepackage{minted}
\usepackage{polyglossia}
\setmainlanguage{russian}
\setotherlanguage{english}
\setkeys{russian}{babelshorthands=true}
\PolyglossiaSetup{russian}{indentfirst=true}
\usepackage{fontspec}
\setmainfont{Liberation Serif}
\usepackage{minted}
\usepackage[left=4cm,right=4cm, top=2cm,bottom=2cm,bindingoffset=0cm]{geometry}
\usepackage{xcolor}
\PassOptionsToPackage{final}{graphicx}
\usepackage{caption}
\usepackage{subcaption}
\usepackage{wrapfig}
\usepackage{array}
\usepackage{multirow}
\usepackage{makecell}
\definecolor{friendlybg}{HTML}{f0f0f0}
\date{}
\title{Задачи на графы}
\hypersetup{
 pdfauthor={},
 pdftitle={Задачи на графы},
 pdfkeywords={},
 pdfsubject={},
 pdfcreator={Emacs 27.1 (Org mode 9.4)}, 
 pdflang={Russian}}
\begin{document}



\section*{Задача 1}
\label{sec:org7b3cc19}
\subsection*{Постановка}
\label{sec:org36b3cf3}

Задан неорентированный граф. Задача состоит в нахождении степени всех
вершин.

\subsection*{Входные данные}
\label{sec:org987b524}

В первой строке число \(t\)
(\(1\leq t \leq 105\)) - количество наборов входных данных.
Далее следуют \(t\) наборов входных данных.

В первой строке каждого набора содержатся два целых числа
\(n\) и \(m\)
(\(1 \leq n \leq 105\), \(0 \leq m \leq 105\)),
где \(n\) — количество вершин в графе, \(m\) — количество рёбер в графе.

В следующих \(m\) строках записаны рёбра, по одному ребру в строке.
Каждое ребро - два числама \(u\) и \(v\) (\(1 \leq u\), \(v \leq n\)),
начало ребра и конец ребра соответственно.

Граф без петель и кратных рёбер.

\subsection*{Выходные данные}
\label{sec:org98b75de}

Для каждого набора в отдельной строке
выведите \(n\) целых чисел, где \(i\text{-е}\) число является степенью \(i\text{-й}\) вершины графа.

\subsection*{Пример}
\label{sec:orgfff544f}

\begin{table}[H]
\begin{center}
\begin{tabular}{|m{4cm}|m{4cm}|}
\hline
Входные данные & Выходные данные \\ \hline
\makecell[l]{
4
\\\\
5 6\\
1 2\\
2 3\\
3 1\\
4 3\\
5 4\\
5 2
\\\\
3 2\\
1 2\\
2 3
\\\\
2 1\\
1 2
\\\\
4 0
}
&
\makecell[l]{
2 3 3 2 2 \\
1 2 1 \\
1 1 \\
0 0 0 0 \\
}
\\ \hline

\end{tabular}
\end{center}
\end{table}

\pagebreak
\section*{Задача 2}
\label{sec:org1ba3041}
\subsection*{Постановка}
\label{sec:orgeac4056}

Постройте \(k\text{-регулярный}\) неориентированный граф из \(n\) вершин.
Если это невозможно, то укажите это.

\subsection*{Входные данные}
\label{sec:orga444191}

В первой строке находится число \(t\)
(\(1 \leq t \leq 1000\)) — количество наборов тестовых данных в тесте.
Далее следуют \(t\) наборов тестовых данных.

Каждый набор состоит из одной строки, в которой записаны
два числа \(n\) и \(k\) (\(1 \leq n,k \leq200\)).

\subsection*{Выходные данные}
\label{sec:org0954c67}

Для каждого набора необходимо вывести:
\begin{itemize}
\item если существуте, то вывести количество рёбер в графе и
ребра в следующих строках.
\item если не существует, то выведите None.
\end{itemize}

\subsection*{Пример}
\label{sec:orgca0865c}

\begin{table}[H]
\begin{center}
\begin{tabular}{|m{4cm}|m{4cm}|}
\hline
Входные данные & Выходные данные \\ \hline
\makecell[l]{
3\\
3 2\\
5 3\\
5 4
}
&
\makecell[l]{
3\\
1 2\\
2 3\\
3 1\\
None\\
10\\
1 2\\
1 3\\
2 3\\
2 4\\
3 4\\
3 5\\
4 5\\
4 1\\
5 1\\
5 2
}
\\ \hline

\end{tabular}
\end{center}
\end{table}

\pagebreak
\section*{Задача 3}
\label{sec:org824f17b}
\subsection*{Постановка}
\label{sec:org3e70d7d}

Постройте наименьший по количеству дуг
непустой ориентированный граф, такой что
степень исхода каждой вершины равна \(d_{1}\), а степень
входа равна \(d_{2}\).

\subsection*{Входные данные}
\label{sec:org0e2dcf0}

В первой строке находится целое число \(t\)
(\(1 \leq t \leq 30\)) — количество наборов
входных данных в тесте.
Далее следуют \(t\) наборов.

В строке каждого набора содержатся
два целых числа \(d_{1}\) и \(d_{2}\) (\(1 \leq d1, d2 \leq 100\)) -
степень исхода и степень входа каждой вершины соответственно.

\subsection*{Выходные данные}
\label{sec:org788f490}

Для каждого набора необходимо вывести:
\begin{itemize}
\item если существуте, то вывести в первой строке Yes,
потом количество вершин и дуг искомого графа.
А в остальных строках пары дуг.
\item если не существует, то выведите None.
\end{itemize}

\subsection*{Пример}
\label{sec:orgb23ed02}

\begin{table}[H]
\begin{center}
\begin{tabular}{|m{4cm}|m{4cm}|}
\hline
Входные данные & Выходные данные \\ \hline
\makecell[l]{
2\\
2 2\\
1 2\\
}
&
\makecell[l]{
Yes\\
2 4\\
1 1\\
1 2\\
2 1\\
2 2\\
None\\
}
\\ \hline

\end{tabular}
\end{center}
\end{table}
\end{document}
