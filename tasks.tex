% Created 2020-10-11 Вс 14:53
% Intended LaTeX compiler: xelatex
\documentclass{extarticle}
                

\usepackage{longtable}
\usepackage{wrapfig}
\usepackage{rotating}
\usepackage[normalem]{ulem}
\usepackage{amsmath}
\usepackage{breqn}
\usepackage{textcomp}
\usepackage{amssymb}
\usepackage{capt-of}
\usepackage{hyperref}
\usepackage{minted}
\usepackage{polyglossia}
\setmainlanguage{russian}
\setotherlanguage{english}
\setkeys{russian}{babelshorthands=true}
\PolyglossiaSetup{russian}{indentfirst=true}
\usepackage{fontspec}
\setmainfont{Liberation Serif}
\usepackage{minted}
\usepackage[left=4cm,right=4cm, top=2cm,bottom=2cm,bindingoffset=0cm]{geometry}
\usepackage{xcolor}
\PassOptionsToPackage{final}{graphicx}
\usepackage{caption}
\usepackage{subcaption}
\usepackage{wrapfig}
\usepackage{array}
\usepackage{multirow}
\usepackage{makecell}
\definecolor{friendlybg}{HTML}{f0f0f0}
\date{}
\title{Задачи на графы}
\hypersetup{
 pdfauthor={},
 pdftitle={Задачи на графы},
 pdfkeywords={},
 pdfsubject={},
 pdfcreator={Emacs 27.1 (Org mode 9.4)}, 
 pdflang={Russian}}
\begin{document}



\section*{Задача 1}
\label{sec:org294f37c}
\subsection*{Постановка}
\label{sec:org41a7803}

Задан неорентированный граф. Задача состоит в нахождении степени всех
вершин.

\subsection*{Входные данные}
\label{sec:org90f3b8c}

В первой строке число \(t\)
(\(1\leq t \leq 105\)) - количество наборов входных данных.
Далее следуют \(t\) наборов входных данных.

В первой строке каждого набора содержатся два целых числа
\(n\) и \(m\)
(\(1 \leq n \leq 105\), \(0 \leq m \leq 105\)),
где \(n\) — количество вершин в графе, \(m\) — количество рёбер в графе.

В следующих \(m\) строках записаны рёбра, по одному ребру в строке.
Каждое ребро - два числама \(u\) и \(v\) (\(1 \leq u\), \(v \leq n\)),
начало ребра и конец ребра соответственно.

Граф без петель и кратных рёбер.

\subsection*{Выходные данные}
\label{sec:orgc9bae44}

Для каждого набора в отдельной строке
выведите \(n\) целых чисел, где \(i\text{-е}\) число является степенью \(i\text{-й}\) вершины графа.

\subsection*{Пример}
\label{sec:orgf80c14d}

\begin{table}[H]
\begin{center}
\begin{tabular}{|m{4cm}|m{4cm}|}
\hline
Входные данные & Выходные данные \\ \hline
\makecell[l]{
4
\\\\
5 6\\
1 2\\
2 3\\
3 1\\
4 3\\
5 4\\
5 2
\\\\
3 2\\
1 2\\
2 3
\\\\
2 1\\
1 2
\\\\
4 0
}
&
\makecell[l]{
2 3 3 2 2 \\
1 2 1 \\
1 1 \\
0 0 0 0 \\
}
\\ \hline

\end{tabular}
\end{center}
\end{table}

\pagebreak
\section*{Задача 2}
\label{sec:org065bf9e}
\subsection*{Постановка}
\label{sec:org69d8a83}

Постройте \(k\text{-регулярный}\) неориентированный граф из \(n\) вершин.
Если это невозможно, то укажите это.

\subsection*{Входные данные}
\label{sec:org5c70f6b}

В первой строке находится число \(t\)
(\(1 \leq t \leq 1000\)) — количество наборов тестовых данных в тесте.
Далее следуют \(t\) наборов тестовых данных.

Каждый набор состоит из одной строки, в которой записаны
два числа \(n\) и \(k\) (\(1 \leq n,k \leq200\)).

\subsection*{Выходные данные}
\label{sec:org27c81e8}

Для каждого набора необходимо вывести:
\begin{itemize}
\item если существуте, то вывести количество рёбер в графе и
ребра в следующих строках.
\item если не существует, то выведите None.
\end{itemize}

\subsection*{Пример}
\label{sec:org3be0cfd}

\begin{table}[H]
\begin{center}
\begin{tabular}{|m{4cm}|m{4cm}|}
\hline
Входные данные & Выходные данные \\ \hline
\makecell[l]{
3\\
3 2\\
5 3\\
5 4
}
&
\makecell[l]{
3\\
1 2\\
2 3\\
3 1\\
None\\
10\\
1 2\\
1 3\\
2 3\\
2 4\\
3 4\\
3 5\\
4 5\\
4 1\\
5 1\\
5 2
}
\\ \hline

\end{tabular}
\end{center}
\end{table}

\pagebreak
\section*{Задача 3}
\label{sec:orgbdf5130}
\subsection*{Постановка}
\label{sec:org2940329}

Постройте наименьший по количеству дуг
непустой ориентированный граф, такой что
степень исхода каждой вершины равна \(d_{1}\), а степень
входа равна \(d_{2}\).

\subsection*{Входные данные}
\label{sec:orgfc35171}

В первой строке находится целое число \(t\)
(\(1 \leq t \leq 30\)) — количество наборов
входных данных в тесте.
Далее следуют \(t\) наборов.

В строке каждого набора содержатся
два целых числа \(d_{1}\) и \(d_{2}\) (\(1 \leq d1, d2 \leq 100\)) -
степень исхода и степень входа каждой вершины соответственно.

\subsection*{Выходные данные}
\label{sec:org4bc246c}

Для каждого набора необходимо вывести:
\begin{itemize}
\item если существуте, то вывести в первой строке Yes,
потом количество вершин и дуг искомого графа.
А в остальных строках пары дуг.
\item если не существует, то выведите None.
\end{itemize}

\subsection*{Пример}
\label{sec:org14b7e61}

\begin{table}[H]
\begin{center}
\begin{tabular}{|m{4cm}|m{4cm}|}
\hline
Входные данные & Выходные данные \\ \hline
\makecell[l]{
2\\
2 2\\
1 2\\
}
&
\makecell[l]{
Yes\\
2 4\\
1 1\\
1 2\\
2 1\\
2 2\\
None\\
}
\\ \hline

\end{tabular}
\end{center}
\end{table}

\pagebreak
\section*{Задача 4}
\label{sec:org1c43528}
\subsection*{Постановка}
\label{sec:orgf9032bf}

Вам заданы неориентированный граф списком его рёбер и множество вершин. Проверьте, что заданные вершины образуют в точности одну или более компонент связности заданного графа.

Иными словами, необходимо проверить можно ли выбрать подмножество компонент связности так, что заданные вершины являются всеми вершинами этого подмножества компонент (и только ими).

\subsection*{Входные данные}
\label{sec:org8bfa450}

В первой строке находится целое число \(t\)
(\(1 \leq t \leq 30\)) — количество наборов
входных данных в тесте.
Далее следуют \(t\) наборов, каждая через пустую строку.

В первой строке примера содержатся количество вершин в графе (\(n\)),
количество рёбер (\(m\)) и количество вершин в множестве (\(k\)).

В следующей строке \(k\) целых чисел - заданное множество вершин.

В следующих \(m\) строках записаны рёбра, по одному ребру в строке.

Граф без петель и кратных рёбер.

\subsection*{Выходные данные}
\label{sec:orgffd6625}

Для каждого набора необходимо вывести:
\begin{itemize}
\item True - если заданные вершину оразуют одну или более компоненту связности.
\item False - в противном случае
\end{itemize}

\subsection*{Пример}
\label{sec:org388f60f}

\begin{table}[H]
\begin{center}
\begin{tabular}{|m{4cm}|m{4cm}|}
\hline
Входные данные & Выходные данные \\ \hline
\makecell[l]{
2
\\\\
4 3 3\\
1 2 3\\
1 2\\
2 3\\
1 3
\\\\
4 2 3\\
1 2 3\\
1 2\\
3 4
}
&
\makecell[l]{
True\\
False
}
\\ \hline

\end{tabular}
\end{center}
\end{table}
\end{document}
